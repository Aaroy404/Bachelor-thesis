%%%%%%%%%%%%%%%%%%%%%%%%%%%%%%%%%%%%%%%%%%%%%%%%%%%%%%%%%%%%%%%%%%%%
%% I, the copyright holder of this work, release this work into the
%% public domain. This applies worldwide. In some countries this may
%% not be legally possible; if so: I grant anyone the right to use
%% this work for any purpose, without any conditions, unless such
%% conditions are required by law.
%%%%%%%%%%%%%%%%%%%%%%%%%%%%%%%%%%%%%%%%%%%%%%%%%%%%%%%%%%%%%%%%%%%%

\documentclass[
  digital,     %% The `digital` option enables the default options for the
               %% digital version of a document. Replace with `printed`
               %% to enable the default options for the printed version
               %% of a document.
%%  color,       %% Uncomment these lines (by removing the %% at the
%%               %% beginning) to use color in the printed version of your
%%               %% document
  oneside,     %% The `oneside` option enables one-sided typesetting,
               %% which is preferred if you are only going to submit a
               %% digital version of your thesis. Replace with `twoside`
               %% for double-sided typesetting if you are planning to
               %% also print your thesis. For double-sided typesetting,
               %% use at least 120 g/m² paper to prevent show-through.
  nosansbold,  %% The `nosansbold` option prevents the use of the
               %% sans-serif type face for bold text. Replace with
               %% `sansbold` to use sans-serif type face for bold text.
  nocolorbold, %% The `nocolorbold` option disables the usage of the
               %% blue color for bold text, instead using black. Replace
               %% with `colorbold` to use blue for bold text.
  lof,         %% The `lof` option prints the List of Figures. Replace
               %% with `nolof` to hide the List of Figures.
  lot,         %% The `lot` option prints the List of Tables. Replace
               %% with `nolot` to hide the List of Tables.
]{fithesis4}
%% The following section sets up the locales used in the thesis.
\usepackage[resetfonts]{cmap} %% We need to load the T2A font encoding
\usepackage[T1,T2A]{fontenc}  %% to use the Cyrillic fonts with Russian texts.
\usepackage[
  main=slovak, %% By using `czech` or `slovak` as the main locale
                %% instead of `english`, you can typeset the thesis
                %% in either Czech or Slovak, respectively.
  english, german, czech, slovak %% The additional keys allow
]{babel}        %% foreign texts to be typeset as follows:
%%
%%   \begin{otherlanguage}{german}  ... \end{otherlanguage}
%%   \begin{otherlanguage}{czech}   ... \end{otherlanguage}
%%   \begin{otherlanguage}{slovak}  ... \end{otherlanguage}
%%
%%
%% The following section sets up the metadata of the thesis.
\thesissetup{
    date        = \the\year/\the\month/\the\day,
    university  = mu,
    faculty     = fi,
    type        = bc,
    department  = Katedra počítačových systémů a komunikací,
    author      = Adam Renčo,
    gender      = m,
    advisor     = {doc. Ing. Michal Brandejs, CSc.},
    title       = {Potvrdenie o absolvovaní školenia s jednoduchým elektronickým podpisom},
    TeXtitle    = {Potvrdenie o absolvovaní školenia s jednoduchým elektronickým podpisom},
    keywords    = {keyword1, keyword2, ...},
    TeXkeywords = {keyword1, keyword2, \ldots},
    abstract    = {%
      This is the abstract of my thesis, which can

      span multiple~paragraphs.
    },
    thanks      = {%
      These are the acknowledgements for my thesis, which can

      span multiple paragraphs.
    },
    bib         = example.bib,
    %% Remove the following line to use the JVS 2018 faculty logo.
    facultyLogo = fithesis-fi,
}
\usepackage{makeidx}      %% The `makeidx` package contains
\makeindex                %% helper commands for index typesetting.
%% These additional packages are used within the document:
\usepackage{paralist} %% Compact list environments
\usepackage{amsmath}  %% Mathematics
\usepackage{amsthm}
\usepackage{amsfonts}
\usepackage{url}      %% Hyperlinks
\usepackage{markdown} %% Lightweight markup
\usepackage{listings} %% Source code highlighting
\lstset{
  basicstyle      = \ttfamily,
  identifierstyle = \color{black},
  keywordstyle    = \color{blue},
  keywordstyle    = {[2]\color{cyan}},
  keywordstyle    = {[3]\color{olive}},
  stringstyle     = \color{teal},
  commentstyle    = \itshape\color{magenta},
  breaklines      = true,
}
\usepackage{floatrow} %% Putting captions above tables
\floatsetup[table]{capposition=top}
\usepackage[babel]{csquotes} %% Context-sensitive quotation marks
\thesisload
\DeclareDelimFormat{multicitedelim}{\addsemicolon\space}
\DefineBibliographyStrings{slovak}{urlfrom = {dostupné z}}
\begin{document}
%% The \chapter* command can be used to produce unnumbered chapters:
\chapter*{Úvod}
\label{chap:intro}
%% Unlike \chapter, \chapter* does not update the headings and does not
%% enter the chapter to the table of contents. I we want correct
%% headings and a table of contents entry, we must add them manually:
\markright{\textsc{Introduction}}
\addcontentsline{toc}{chapter}{Introduction}

Jednou z dôležitých požiadaviek na prevádzku školení bezpečnosti a ochrany zdravia pri práci (ďalej len BOZP) a požiarnej ochrany (ďalej len PO) je udržiavanie dokumentácie o ich absolvovaní frekventantmi.

Z právnych dôvodov musí byť preukázateľné, že bol daný frekventant korektne zaškolený vo všetkých preňho relevantných smeroch. Aby bola takáto dokumentácia právne nevyvrátiteľná, musí byť potvrdená ako absolventom školenia, tak aj jeho školiteľom. Informačný systém Masarykovej univerzity (IS MU) používa na toto potvrdenie záznamov fyzický podpis oboch entít.

Cieľom tejto práce je previezť proces potvrdzovania dokumentácie o absolvovaní školení na Masarykovej univerzite (MU) do online formy. Za týmto účelom je nevyhnutná implementácia potvrdení s jednoduchým elektronickým podpisom. Súčasťou práce je aj následný prevod spisov tejto dokumentácie do elektronickej formy. Tieto zmeny uľahčia prácu všetkým zúčastneným stranám prevádzky školení.

Najprv som preštudoval právne (kapitola~\hyperref[kap-1]{1}) a používateľské (kapitola~\hyperref[kap-1]{2}) požiadavky na udržiavanú dokumentáciu. Následne som upravil funkcionalitu tvorby záznamov tak, aby odpovedala novej forme využívajúcej elektronický podpis (kapitola~\hyperref[kap-1]{3}). Ďalej som pridal možnosť samotného elektronického podpisu a zapečatenia pečaťou IS MU v aplikácii Úřadovna (kapitola~\hyperref[kap-1]{4}). Ako posledné bolo potrebné spojiť nové dokumenty s ich relevantnými elektronickými spismi v aplikácii Úschovna (kapitola~\hyperref[kap-1]{5}).

Výsledné úpravy systému umožňujú frekventantom potvrdiť absolvovanie školenia priamo v IS MU v aplikácii BOZP a zaručujú potvrdenie školiteľa, ktorým je po týchto úpravách IS MU. Nové potvrdenia o absolvovaní školenia podpísané elektronicky spĺňajú všetky požiadavky na spravovanie dokumentácie BOZP a PO.


\chapter{Školenia BOZP a PO}
\label{kap-1}
V tejto práci sú použité pojmy označujúce osoby, ktoré majú spojenie so školeniami BOZP a PO. Kvôli prehľadnosti ich v tejto kapitole definujeme:

~

    \textbf{Frekventant} je pojem označujúci osobu, ktorá má povinnosť splniť školenie.~\cite[9]{kandova2019}\\
    
    \textbf{Školiteľ} predstavuje entitu (osobu alebo systém), ktorá frekventanta preškoľuje.\\
    
    \textbf{Absolvent} školenia je frekventant, ktorý splnil dané školenie.\\

    \textbf{Správca dokumentácie} je osoba poverená uchovávaním dokumentácie o danom školení.
~
\\\\
Náplň tejto práce je priamo spojená s témou školení BOZP a PO. Z tohto dôvodu je nevyhnutné zhrnúť pár základných informácií na túto tému, ktoré sú dôležité pre porozumenie problémov riešených v nasledujúcich kapitolách.

Zákonník práce Českej republiky nariaďuje zamestnávateľom poskytovať zamestnancom školenia BOZP~\cite[§~103~odst.~3]{cesko_zakonik_prace}, PO~\cite[§~349~odst.~1]{cesko_zakonik_prace} a školenie prvej pomoci~\cite[§~102~odst.~6]{cesko_zakonik_prace}. Zámerom týchto školení je informovanie zamestnancov o možných rizikách, na ktoré môžu pri práci naraziť, a naučenie základných zručností potrebných pre minimalizovanie úrazov a škôd spôsobených v prípade stretu s rizikovou situáciou.

Zákonník ďalej nariaďuje, aby boli frekventanti periodicky preškoľovaní. Frekvenciu si zamestnávateľ stanovuje sám, ale zákon nariaďuje minimum dvoch rokov pre nevedúcich zamestnancov a troch rokov pre vedúcich zamestnancov.~\cite[~§103~odst.~3]{cesko_zakonik_prace}

\section{Variabilita školení}
Presný obsah školení nie je zákonom definovaný. Definované sú iba záchytné body, ktoré sú v školení nevyhnutné. Príkladom je nariadenie, aby školenie PO zahŕňalo požiarny poriadok, požiarne poplachové smernice, prípadne evakuačný plán a ďalšiu dokumentáciu obsahujúcu stanovenie podmienok PO daného pracoviska.~\cite[~§23~odst.~1~c)]{cesko_vyhlaska_poziarna_prevence}. Presný obsah tejto dokumentácie je ale v každom školení upravený tak, aby bol relevantný vzhľadom na pracovisko, na ktorom sa školenie prevádzkuje.

K väčšiemu počtu variánt školení prispieva aj fakt, že školenia PO musia byť špecificky upravené pre vedúcich zamestnancov a pre ostatných zamestnancov.~\cite[~§16~odst.~2]{cesko_zakon_pozarni_ochrana} Častokrát tak vznikajú dve verzie k školeniam viazaným na rovnaké pracovisko a pracovnú náplň.

Varianty školení môžu byť označené svojim číslom osnovy. Toto číslo zamestnávateľom obvykle dodáva odborný špecialista BOZP. Pomocou čísla osnovy je následne možné určiť, akým obsahom bol frekventant preškolený. Z tohto dôvodu sa označenie pridáva do záznamu o školení, ktorý frekventant a školiteľ podpisujú.

\section{Dokumentácia školení}
Zamestnávateľ je od zákona povinný udržiavať dokumentáciu o tom, že frekventanti absolvovali školenia BOZP a PO. ~\cites[§~103~odst.~3]{cesko_zakonik_prace}[§~27]{cesko_vyhlaska_poziarna_prevence}

Táto dokumentácia je následne používaná v právnych procesoch v prípade úrazu alebo škody na pracovisku. Z tohto dôvodu je dôležité, aby boli záznamy dobre uchovávané a aby obsahovali všetky relevantné informácie. Zákon presné znenie záznamov síce nešpecifikuje, no je odporúčané, aby obsahovali aspoň identifikačné údaje absolventa a školiteľa, informácie o absolvovanom školení a dátum absolvovania.~\cite{prevent_bozp}

\chapter{Školenia v IS MU}
\label{kap-2}
Práca je zameraná na úpravy systému školení v IS MU, preto v nasledujúcej kapitole skrátene popíšeme, ako sú tieto školenia momentálne prevádzkované. 

\section{Aplikácia Bozp}
Frekventanti sa k absolvovaniu školenia dostanú pomocou aplikácie Bozp.
\footnote{\url{https://is.muni.cz/auth/bozp/}} V aplikácii Bozp sú školenia tvorené elektronickými kurzami, ktoré môžu obsahovať jednu alebo viac z nasledujúcich častí~\cite[19]{kandova2019}:
\begin{enumerate}
    \item \textbf{Prezentácia} – obsahuje hlavný výklad a vizuálne materiály.
    \item \textbf{Súbory na stiahnutie} – textové dokumenty, ktoré si účastníci musia stiahnuť a prečítať.
    \item \textbf{Otázky} – test, ktorý musia účastníci správne zodpovedať na overenie porozumenia školenia.
\end{enumerate}

\section{Varianty školení a kurzov}
Občas sú v školeniach rôzne verzie kurzov a frekventant si môže vybrať, ktorý kurz absolvuje. Ak je potrebné školenie realizovať vo viacerých jazykoch, pre každú jazykovú variantu sa vytvorí samostatný kurz. Pre úspešné absolvovanie školenia stačí absolvovať ktorýkoľvek kurz patriaci pod toto školenie.

Varianty celých školení majú obvykle svoje číslo osnovy. Občas ale číslo osnovy so školením dodané nie je. Potom je dôležité uchovávať skrátený obsah školenia, ktorý osnovu nahradzuje. 

IS MU obsahuje v čase písania tejto práce viac ako 100 variant školení BOZP a PO a približne 130 kurzov, ktoré tieto školenia realizujú.

Toto vysoké množstvo variánt je neskôr dôležité v problematike dokumentácie absolvovaných školení, ktorá je riešená v ďalších kapitolách práce.

POCET PREJDENYCH SKOLENI, CO TO ZNAMENA PRE DOKUMENTACIU


\chapter{Dokumentácia školení v IS MU}
\label{kap-3}
V IS MU sa dokumentácia školení uchováva pomocou \textbf{Záznamov o absolvovaní školení} (ďalej len Záznam). Sú to systémom generované dokumenty formátu PDF. Záznam je definovaný ako dokument, ktorý potvrdzuje, že školiteľ v určený dátum preškolil frekventanta obsahom školení, na ktoré sa Záznam vzťahuje. Pre plnú platnosť Záznamu je potrebné ho vytlačiť, fyzicky podpísať absolventom aj školiteľom a odovzdať správcovi dokumentácie.

Záznamy obsahujú nasledujúce informácie:

\begin{markdown}
  * Informácie o&nbsp;relevantných absolvovaných školeniach;
  * Dátum absolvovania školenia;
  * Identifikačné údaje absolventa;
  * Identifikačné údaje školiteľa (vedúceho zamestnanca);
  * Priestor na podpis absolventa a&nbsp;školiteľa;
  * Meno osoby, ktorej sa záznam odovzdáva (správca dokumentácie).
\end{markdown}

\section{Podpis a odovzdávanie Záznamov}
Keďže záznamy vyžadujú podpis fyzickej osoby ako školiteľa, je potrebné túto osobu určiť. Ak nie je vedením špecifikovaná osoba zodpovedná za školenie zamestnancov, je touto povinnosťou poverený priamy nadriadený frekventanta školenia. Záznam o absolvovaní školenia následne musí táto osoba podpísať, čím potvrdzuje, že školenie skutočne prebehlo. V praxi väčšina školení nemá špecifikovanú zodpovednú osobu, a preto sa ďalej budeme zaoberať najmä touto variantou.

Záznamy o všeobecných školeniach, ako napríklad základné školenie BOZP alebo školenie prvej pomoci, podpisujú nadriadený zamestnanci všetkých \textit{kmeňových pracovísk}\footnote{Kmeňové pracovisko - miesto alebo organizačná jednotka, ku ktorej je pracovník formálne pridelený ako svoje hlavné pracovné miesto.} absolventa. Jedna osoba môže mať kmeňových pracovísk na MU akýkoľvek počet.

Záznamy o špecifických školeniach podpisujú nadriadení zamestnanci iba na pracoviskách (aj iných, ako kmeňových) absolventa, ktoré dané školenie vyžadujú.

Podpísaný Záznam musí absolvent následne odovzdať správcovi dokumentácie. V mnohých prípadoch túto úlohu plní samotný školiteľ, avšak pri každom preškolení je možné presne určiť osobu zodpovednú za správu dokumentácie.

Môže nastať situácia, že zamestnanec školenie absolvuje a počas jeho platnosti vznikne zamestnancovi nový úväzok na inom pracovisku MU, ktoré toto školenie taktiež požaduje. V tomto prípade vedenie MU požiadalo, aby miesto opakovaného preškoľovania stačilo Záznam o školení dať podpísať školiteľovi (nadriadenému zamestnancovi) na novom pracovisku. Preto musia byť všetky potrebné informácie na generovanie nového Záznamu s iným menom školiteľa uchovávané po celú dobu platnosti školenia.

\section{Skupiny školení v Záznamoch}
Záznamy vyžadujú veľké množstvo administratívnych úkonov spojených s ich uchovávaním. Preto bola na Záznamy stanovená požiadavka, aby mohli obsahovať viacero školení, čím sa ich počet zníži.

Niektoré školenia ale nemôžu spolu patriť do jedného Záznamu. Separátne musia byť napríklad obecné a špecifické školenia, pretože ich treba odovzdávať na iné pracoviská a iným správcom dokumentácie. Provozní odboru MU zároveň požiadal, aby školenie pre vodičov nebolo na rovnakom Zozname ako školenia BOZP a PO. Preto sú školenia rozdelené do takzvaných záznamových skupín, ktoré určujú, či môžu byť dve školenia zahrnuté do rovnakého Záznamu.

\section{Problémy so Záznamami}
Záznamy sú síce z právneho hľadiska plne platné, ale spôsob ich generácie a uchovávania so sebou prináša veľké množstvo komplikácií, ktoré ich robia neefektívnymi.

\subsection*{Fyzická forma}
Jednou z najvýraznejších nevýhod je ich fyzická forma. Aj napriek snahám o zjednodušenie ich správy, ako je napríklad spájanie školení do skupín, vyžadujú od absolventov, školiteľov a správcov dokumentácie mnohé úkony, ktoré moderná digitálna forma dokumentácie výrazne zjednodušuje alebo úplne eliminuje.

Častokrát sa napríklad stáva, že zamestnanci Záznam zabudnú odovzdať, odovzdajú nesprávnej osobe, alebo Záznam vôbec ani nevytlačia. Ak by Záznamy boli držané čisto digitálne, všetky tieto akcie by mohli byť automatizované a zjednodušené. Systém by zároveň vedel ľahšie kontrolovať, ktoré akcie už boli vykonané, a upozorniť relevantnú osobu v prípade, že niečo zostalo nedokončené.

Ďalším príkladom je potreba fyzických spisov. Systém neobsahuje úložisko vytlačených Záznamov, a preto sú Záznamy udržiavané iba vo fyzických spisoch správcami dokumentácie. Táto potreba by taktiež bola digitalizáciou úplne odstránená.

\subsection*{Podpis Záznamu na novom pracovisku}
Ak zamestnanec nastúpi na nové pracovisko, ale niektoré požadované školenie už má absolvované a platné, nie je ním preškolený znova. Namiesto školenia mu nový nadriadený zamestnanec podpíše Záznam o školení ako jeho školiteľ.

Komplikáciu predstavuje fakt, že Záznam z definície tvrdí, že daný školiteľ preškolil frekventanta daným školením v určenom dátume. Ak je na Zázname nové dátum školenia (po vzniku nového úväzku), tak je toto tvrdenie nepravdivé, pretože ku žiadnemu preškoleniu neprišlo. Takisto však nový nadriadený zamestnanec nemôže podpísať Záznam o školení s dátumom, v ktorom zamestnanec školenie reálne absolvoval, pretože v danej dobe ešte nebol nadriadeným tohto zamestnanca.

\subsection*{Dynamická generácia}
Za určitých okolností je nutné Záznam generovať nanovo. Nová generácia je potrebná najmä pri zmenách, ako sú napríklad pridanie školenia rovnakej Záznamovej skupiny školení alebo prestup na nové pracovisko. Táto potreba so sebou prináša problémy s udržiavaním histórie dát.

Keďže sa v digitálnej forme Záznamy neuchovávajú, musí byť každý Záznam generovaný nanovo. V prípade, že sa zmení legislatíva alebo iné parametre, ktoré sú súčasťou Záznamov, môže nastať problém so spiatočnou generáciou. Záznam vygenerovaný pre školenie absolvované pred zmenami by totiž nemusel predstavovať korektnú dokumentáciu o jeho absolvovaní.

\section{Zvažované úpravy}
Veľké množstvo problémov Záznamov je úzko späté s ich fyzickou formou. Preto je hlavným zvažovaným riešením pre IS MU prevod tejto dokumentácie do digitálnej formy, v ktorej by mohli byť procesy jej prevádzky zjednodušené a automatizované.


\chapter{Elektronická identifikácia a dôveryhodné služby}
Nariadenia Európskej únie (ďalej len EÚ) č. 910/2014 a č. 2024/1183 (citované v ich konsolidovanej verzii) o elektronickej identifikácii a dôveryhodných službách pre elektronické transakcie na vnútornom trhu (ďalej len eIDAS) definujú kľúčové pojmy týkajúce sa elektronických podpisov a elektronických pečatí. V nariadeniach sú tieto pojmy vyložené s cieľom zabezpečiť právnu istotu a interoperabilitu elektronických podpisov v rámci členských štátov EÚ.~\cite{eidas2024} Pre správne porozumenie legislatívnych požiadaviek týkajúcich sa elektronických identifikačných prostriedkov je nevyhnutné jasne vymedziť pojmy použité v týchto dokumentoch.

\textbf{Elektronický podpis} je v eIDAS definovaný ako akékoľvek dáta v elektronickej podobe, ktoré sú pripojené k iným dátam v elektronickej podobe, alebo sú s nimi logicky spojené a podpisujúca osoba ich používa k podpísaniu.~\cite[čl.~3,~odst.~10]{eidas2024} Elektronický podpis nesmie byť zamietnutý ako dôkaz a nesmú mu byť odopreté právne účinky čisto z dôvodu, že má elektronickú podobu alebo nespĺňa požiadavky kvalifikovaného elektronického podpisu.~\cite[čl.~25,~odst.~1]{eidas2024}

Pod pojmom \textbf{fyzická osoba} sa rozumie ľudská bytosť, ktorá je právnym subjektom schopným využívať elektronické identifikačné a dôveryhodné služby. \textbf{Podpisujúca osoba} je definovaná ako fyzická osoba, ktorá vytvára elektronický podpis.~\cite[čl.~3,~odst.~9]{eidas2024}

\textbf{Elektronická pečať} sú elektronické dáta priamo alebo logicky pripojené k iným dátam v elektronickej podobe za účelom zachovania ich pôvodu a integrity.~\cite[čl.~3,~odst.~25]{eidas2024}

Pojem \textbf{právnická osoba} označuje akúkoľvek entitu, ako je spoločnosť, združenie alebo organizácia, ktorá má právnu spôsobilosť a je uznávaná právom. \textbf{Pečatiaca osoba} je právnická osoba, ktorá vytvára elektronickú pečať.[čl.~3,~odst.~9]{eidas2024}

\textbf{Dáta pre vytváranie elektronických podpisov} sú jedinečné dáta, pomocou ktorých podpisujúca osoba vytvára svoje elektronické podpisy.~\cite[čl.~3,~odst.~13]{eidas2024}

\textbf{Prostriedok pre vytváranie elektronických podpisov} je programové vybavenie alebo technické zariadenie, ktoré sa používa pre vytváranie elektronických podpisov.\cite[čl.~3,~odst.~22]{eidas2024} \textbf{Kvalifikovaný prostriedok pre vytváranie elektronických podpisov} je prostriedok pre vytváranie elektronických podpisov, ktorý spĺňa nasledujúce podmienky~\cite[čl.~3,~odst.~23]{eidas2024}:

\begin{itemize}
  \item Zaisťuje dôvernosť dát pre vytváranie elektronických podpisov, ktoré boli použité pri vytváraní elektronického podpisu;
  \item zaisťuje, aby dáta pre vytváranie elektronických podpisov použité pre vytváranie elektronického podpisu mohli byť prakticky použité iba raz;
  \item zaisťuje neodvoditeľnosť dát pre vytváranie elektronických podpisov použitých pre vytváranie elektronického podpisu~a ochranu elektronického podpisu v súčasnosti dostupnými technickými prostriedkami proti falšovaniu;
  \item zaisťuje, že podpisujúca osoba mala možnosť spoľahlivo ochrániť dáta pre vytváranie elektronických podpisov použité pre vytváranie elektronického podpisu pred ich zneužitím inou osobou; a
  \item nemení podpisované dáta~a nebráni tomu, aby tieto dáta boli pred podpísaním predložené podpisujúcej osobe.~\cite[príloha II]{eidas2024}
\end{itemize}

Prostriedkom pre vytváranie elektronických podpisov sú napríklad aplikácie umožňujúce tvorenie elektronických podpisov, hardvérové zariadenia obsahujúce dáta pre vytváranie elektronických podpisov alebo biometrické zariadenia používané pre potvrdenie identity podpisujúcej osoby. 

Aby bol elektronický podpis považovaný za \textbf{zaručený elektronický podpis}, musí spĺňať nasledujúce podmienky~\cite[čl.~3,~odst.~11]{eidas2024}:

\begin{itemize}
    \item Je jednoznačne spojený s podpisujúcou osobou;
    \item umožňuje identifikáciu podpisujúcej osoby;
    \item je vytvorený pomocou dát pre vytváranie elektronických podpisov a podpisujúca osoba tieto dáta dokáže s vysokou úrovňou dôvery použiť pod svojou výhradnou kontrolou; a
    \item je pripojený k dátam, ktoré podpisuje, takým spôsobom, aby bolo možné zistiť ich akúkoľvek následnú zmenu.~\cite[čl.~26]{eidas2024}
\end{itemize}

\textbf{Dáta pre overovanie platnosti} sú dáta používané k overeniu platnosti elektronického podpisu alebo elektronickej pečate.~\cite[čl.~3,~odst.~40]{eidas2024}

Za \textbf{službu vytvárajúcu dôveru} sa považuje elektronická služba poskytovaná za finančnú odmenu a zabezpečuje rôzne činnosti súvisiace s elektronickými podpismi a ďalšími službami, ktoré sú nevyhnutné na zaručenie dôvery pri elektronických transakciách.~\cite[čl.~3,~odst.~16]{eidas2024} \textbf{Kvalifikovanou službou vytvárajúcou dôveru} je služba vytvárajúca dôveru, ktorá splňuje použiteľné požiadavky v nariadení eIDAS.~\cite[čl.~3,~odst.~17]{eidas2024}

Pojem \textbf{Poskytovateľ služieb vytvárajúcich dôveru} označuje akúkoľvek fyzickú alebo právnickú osobu, ktorá poskytuje službu vytvárajúcu dôveru.~\cite[čl.~3,~odst.~19]{eidas2024} \textbf{Kvalifikovaný poskytovateľ služieb vytvárajúcich dôveru} je poskytovateľ služieb vytvárajúcich dôveru, ktorý poskytuje kvalifikovanú službu vytvárajúcu dôveru a ktorému bol udelený status kvalifikovaného poskytovateľa orgánom dohľadu.~\cite[čl.~3,~odst.~20]{eidas2024}

\textbf{Certifikát per elektronický podpis} je elektronické potvrdenie, ktoré spojuje dáta pre overovanie platnosti podpisu s fyzickou osobou a potvrdzuje minimálne jej meno alebo pseudonym\footnote{Pseudonym - fiktívne meno používané osobou na skrytie svojej pravej identity}.~\cite[čl.~3,~odst.~14]{eidas2024} \textbf{Kvalifikovaným certifikátom pre elektronický podpis} je každý certifikát pre elektronický podpis, ktorý je vydaný kvalifikovaným poskytovateľom služieb vytvárajúcich dôveru a obsahuje všetky informácie požadované v prílohe I nariadenia eIDAS.~\cites[čl.~3,~odst.~15]{eidas2024}[príloha I]{eidas2024}

\textbf{Kvalifikovaný elektronický podpis} je zaručený elektronický podpis, ktorý je vytvorený kvalifikovaným prostriedkom pre vytváranie elektronických podpisov a založený na kvalifikovanom certifikáte pre elektronické podpisy.

\chapter{Úradovňa}
Všetky návrhy, ktoré budú ďalej v práci preberané sú založené na prepojení aplikácie Bozp so sadou aplikáciou Úradovňa.\footnote{\url{https://is.muni.cz/auth/uradovna/}} Úradovňa v sebe zahŕňa kompletnú spisovú službu, ktorá eviduje komunikáciu medzi školou a ostatnými subjektami, a evidenciu postupu riešenia daných vecí. Ponúka aj možnosť automatickej kontroly, automatického doplňovania údajov a ich hromadného zadávania.~\cite{uradovna2024}

Úradovňa je primárne určená na ukladanie a správu údajov v elektronickej podobe. K týmto elektronicky evidovaným dátam je ale možné pripojiť aj dokumenty vo fyzickej podobe vrátane naskenovaných verzií dokumentov.~\cite{uradovna2024}

V Úradovni je základnou jednotkou spis. Táto jednotka analogicky k papierovým spisom obsahuje všetky relevantné informácie o určitej záležitosti. K spisu môžu byť pridané atribúty pre upresnenie osôb a vecí, ktorých sa týka. Každý spis sa skladá z úkonov a patrí do jednej agendy.~\cite{uradovna2024}

Úkony predstavujú údaje držané v danom spise. Takisto môžu obsahovať upresňujúce atribúty.~\cite{uradovna2024} Úkonom môže byť napríklad založenie spisu, vloženie dokumentu, ale aj pre túto prácu relevantnejšie podpísanie dokumentu a zapečatenie dokumentu.

Agendy sú tvorené spismi, ktoré evidujú záležitosti rovnakého charakteru. Nastavenie agendy určuje, aké atribúty majú spisy a úkony, ktoré do nej spadajú. Každá agenda je následne priradená na správu konkrétnemu pracovisku.~\cite{uradovna2024}

Úradovňa takisto ponúka aj nastavenia prístupových práv k udržiavaným dokumentom. Existuje implicitný a explicitný druh práv. Implicitné práva na prístup vznikajú pre osobu, ktorej sa spis týka, a zároveň pre všetky osoby, od ktorých je žiadaný úkon do daného spisu. Explicitné práva sú potrebné pre administratívnu stránku držania dokumentácie. V kontexte dokumentácie školení je najdôležitejším právo j\_spis. Toto právo je viazané na pracovisko a umožňuje čítanie spisov a pridávanie úkonov v jeho agendách.~\cite{uradovna2024}. Pomocou tohto práva je tak možné správcom dokumentácie sprístupniť spisy zamestnancov, za ktorých sú zodpovední.

\chapter{Návrh možností}
\label{kap-4}
Záznamy ako dokumentácia absolvovaných školení so sebou zjavne prinášajú veľké množstvo problémov. Hlavnou motiváciou tejto práce je práve úprava systému tak, aby boli tieto problémy minimalizované alebo úplne odstránené. Za týmto účelom sme preskúmali rôzne možnosti úprav aplikácií v rámci IS MU, ktoré podrobnejšie rozoberieme v tejto kapitole.

 

\section{Presun Záznamov do aplikácie Úradovňa}
Pôvodný zámer správy Záznamov bol, že po ich vytlačení budú v papierovej forme podpísané školiteľom aj absolventom a následne naskenované a vložené do digitálneho spisu absolventa. Riešenie by využilo funkcionalitu Úradovne pre spracovávanie papierových dokumentov do spisov.

Záznamy by bolo potrebné označiť, aby sa dali efektívne skenovať a digitálne skladovať. Skenovanie dokumentov v Úradovni funguje pomocou čiarových kódov~\cite{uradovna2024}, a preto by v aplikácii Bozp bolo potrebné implementovať podporu pre tvorbu čiarových kódov, ktoré by boli rozpoznateľné v oboch aplikáciách. Touto úpravou by vznikli elektronické spisy o školeniach, ktoré sú prehľadnejšie a efektívnejšie ako tie fyzické.

Ostatné problémy Záznamov by táto úprava ale vôbec nevyriešila. Záznamy by stále požadovali vytlačenie do papierovej formy, podpísanie školiteľa aj absolventa a vznikla by potreba Záznam naskenovať miesto odovzdania správcovi dokumentácie. 

Ďalším závažným problémom riešenia je overenie obsahu skenovaného dokumentu. Spisová služba Úradovne dokáže rozoznať čiarový kód a dokument zaradiť k relevantnému spisu. Nemá ale ako overiť, že daný naskenovaný dokument obsahuje presne to, čo o jeho obsahu tvrdí jeho čiarový kód. Do spisov by sa tak mohli dostať nesprávne alebo falošné dokumenty v prípade, že by boli označené čiarovým kódom pre tento spis.

\section{Digitalizovanie Záznamov a podpis v Úradovni}
Následne vznikol návrh Záznamy previezť z fyzickej formy do digitálnej a udržiavať ich v Úradovni. Po absolvovaní školenia by bol Záznam digitálne presunutý do spisu absolventa a následne elektronicky podpísaný školiteľom a absolventom.

Oproti predošlému návrhu by tak neboli potrebné žiadne administratívne úkony s fyzickými dokumentami. Správcom dokumentácie by sa v Úradovni ľahšie kontrolovalo, či sú Záznamy podpísané oboma osobami. Zároveň by nebolo Záznamy potrebné skenovať, takže by riešenie eliminovalo problém s overovaním obsahu skenovaného dokumentu. Aplikácia Bozp by iba finálny Záznam interne predala do Úradovne. 

Takáto úprava ale stále nerieši ostatné problémy Záznamov. V prípade úväzkov na viacerých pracoviskách by vznikali komplikácie s vznikom viacerých Záznamov k jednému školeniu. Muselo by vzniknúť viacero rozdielnych Záznamov pre rovnaké školenia a pre každý Záznam by bolo potrebné založiť separátny spis na konkrétnom pracovisku absolventa. Rovnako pretrvávajú aj problémy určovania školiteľa a správcu dokumentácie pre daný Záznam. Vyriešený nie je ani problém dynamickej generácie Záznamov. 

  \printbibliography[heading=bibintoc] %% Print the bibliography.

\end{document}
